\documentclass[]{article}

\usepackage{amsmath}
\usepackage{amsfonts}
\usepackage{amssymb}
\usepackage[polish]{babel}
\usepackage{listings}
\usepackage[T1]{fontenc}

%opening

\title{}
\author{}

\begin{document}

\tableofcontents
\pagebreak

\section{Dostępne funkcje}
	\begin{lstlisting}[language=Python, gobble=12]
		Initiator().create() -> SimFrame
			
		Simulator().simulate(arg: SimFrame) -> SimFrame
		
		SimFrame::get_particles() -> List[Particle]
		
		Particle::get_position() -> (float, float)
		
		Particle::get_velocity() -> (float, float)
	\end{lstlisting}
\pagebreak

\section{Użycie}
	\subsection{Initiator}
		Wywołuje się go jak powyżej, zwraca początkowy stan świata jako obiekt typy SimFrame
		
	\subsection{Simulator}
		Wywołuje się go jak wyżej, funkcja simulate przyjumuje jeden stan świata i przekształca go w drugi, jako rezultat upływu czasu
		
	\subsection{SimFrame}
		Na obeikcie typu SimFrame można użyć metody get particles, która zwróci tablicę obiektów typy Particle, które będą odpowiadać wszystkom cząsteczkom i ich stanom w danym momencie czasu opisywanym przez SimFrame
		
	\subsection{Particle}
		\subsubsection{get position}
			Tą metode można wywołać na obiekcie typu Particle i zwróci pozycję cząsteczki jako dwójkę, gdzie pierwszy element to koordynat x a drugi y
		\subsubsection{get velocity}
			Tą metode można wywołać na obiekcie typu Particle i zwróci prędkość cząsteczki jako dwójkę, gdzie pierwszy element to składowa x a druga y

\pagebreak

\section{Testowanie}
	\subsection{Makrostany}
		\begin{lstlisting}
			python3 main.py -m
		\end{lstlisting}
	\subsection{Prawdopodobieństwo termodynamiczne}
		\begin{lstlisting}
			python3 main.py -t
		\end{lstlisting}
	\subsection{Entropia}
		\begin{lstlisting}
			python3 main.py -e
		\end{lstlisting}
	\subsection{Entropia z wykresem}
		\begin{lstlisting}
			python3 main.py -c nazwa_pliku_do_zapisu
		\end{lstlisting}
	\subsection{Ogólny help}
		\begin{lstlisting}
			python3 main.py -h
		\end{lstlisting}
	
\end{document}
