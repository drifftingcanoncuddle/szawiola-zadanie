\documentclass[]{article}

\usepackage{amsmath}
\usepackage{amsfonts}
\usepackage{amssymb}
\usepackage[polish]{babel}
\usepackage{listings}
\usepackage[T1]{fontenc}

%opening

\title{}
\author{}

\begin{document}

\tableofcontents
\pagebreak

\section{Dostępne funkcje}
	\begin{lstlisting}[language=Python, gobble=12]
		Initiator().create() -> SimFrame
			
		Simulator().simulate(arg: SimFrame) -> SimFrame
		
		SimFrame::get_particles() -> List[Particle]
		
		Particle::get_position() -> (float, float)
		
		Particle::get_velocity() -> (float, float)
		
		ConfigMainpulator().read(arg:ConfigFields) -> str
	\end{lstlisting}
\pagebreak

\section{Użycie}

	\subsection{SimFrame}
		Na obiekcie typu SimFrame można użyć metody get.particles, która zwróci tablicę obiektów typu Particle, które będą odpowiadać wszystkim cząsteczkom i ich stanom w danym momencie czasu opisywanym przez SimFrame
		
	\subsection{Particle}
		\subsubsection{get.position}
			Tą metode można wywołać na obiekcie typu Particle i zwróci pozycję cząsteczki jako dwójkę, gdzie pierwszy element to koordynat x a drugi y
		\subsubsection{get.velocity}
			Tą metode można wywołać na obiekcie typu Particle i zwróci prędkość cząsteczki jako dwójkę, gdzie pierwszy element to składowa x a druga y
	
	\subsection{ConfigManipulator}
		Służy do czytania z configu, posiada jedna publczną metodę - read. Przyjmuje ona jedno z mozliwości:
		\begin{lstlisting}[language=Python, gobble = 20]
			size # wielkosc boku pudelka w ktorym jest animacja
			maxSpeed # maksymalna wypadkowa predkosc czasteczki 
			particleAmount # ilosc czasteczek
			boxSize # wielkosc pojedynczej komorki - do liczenia mikrostanow, entropii itd.
			time # calkowita ilosc kalatek do wyrenderowania
			timeDelta # dyskretny krok czasu
			init_state_file # plik  z ustawieniami poczatkowymi
			particleSize # wielkosc czasteczki
			maximalDistanceAsCollision # do ustawien silnika
			maximalTimeDeltaAsColliding # jw
		\end{lstlisting}
		Używa się tego tak:
		\begin{lstlisting}[gobble = 20]
			ConfigManipulator().read(ConfigFields.cos_z_powyzszych)
		\end{lstlisting}
		
	\subsection{Plik simulation.sim}
		Jest to zapicklowana tablica z kolejnymi SimFrame'ami, jest ich tam tyle na ile został ustawiony time w pliku $config.ini$. Aby ją odpicklować można użyć metody $pickle.load(nazwa pliku)$
\pagebreak

\section{Testowanie}
	\subsection{Uruchamianie}
		\begin{lstlisting}[language=bash, gobble=20]
			python3 -s # tworzy plik simulation.sim
			python3 -r # odtwarza plik konfiguracyjny
			python3 -f # wczytuje poczatkowe polozenie czasteczek z pliku
			python3 -p # tworzy wykres do zadania 1
			python3 -v # zadanie 2
			python3 -t # tworzy wykres do prawdopodobienstwa termodynamicznego
			python3 -e # tworzy wykres z entropia
			python3 -h # ogolny help w bashu
			
		\end{lstlisting}
	\subsection{config.ini}
		Można sobie w nim wszystko pozmieniać, a jak coś pójdzie nie tak to można użyć flagi r i wszystko się przywróci do stanu początkowego
		
\section{TODO}
	Wszystkie klasy po wywołaniu funkcji evaluate() muszą zczytać z pliku simulate.sim dane i stworzyć animacje
	
	\paragraph{DrawVelocity} Tworzy animację położenia cząsteczek w przestrzeni VxVy, gdzies na wykresie powinna być informacja jaki numer klatki się aktualnie wyświetla i zapisuje ją do pliku .mp4
	\paragraph{DrawTHPrb} Tworzy wykres zmian prawdopodobienstwa i zapisuje go do pliku
	\paragraph{DrawEntropy} Tworzy wykres zmian entropii i zapisuje go do pliku
	
\section{Przyblizony sposób używania}
	\begin{enumerate}
		\item Uzytkownik tworzy symulacje flagą -s
		\item Uruchamia do woli różne inne dostępne flagi
		\item Tworzy inna symulację flagą -s
	\end{enumerate}	
\end{document}
